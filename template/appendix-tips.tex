\chapter{Tips and Tricks}
\label{appendix:tips}

This appendix lists a few considerations for modifying the template.

\begin{itemize}
\item The margin dimensions are set at the top of \texttt{utdiss.tex}. If you decide to change them, add \verb"\usepackage{layout}" to \texttt{config.tex} and use the \verb"\layout*" command in the body of the document to see a schematic of the page setup.
See \Cref{figure:layout}.
\index{layout}

\item This template uses the \texttt{cleveref} package to reference equations and sections.
Instead of \verb"\ref{}" or \verb"\eqref{}", use \verb"\cref{}".
Use \verb"\Cref{}" if you want to ensure that the first word is capitalized (e.g., `Figure').
\index{commands!cref@\verb+\cref{}+}

\item This is how a labeled equation looks in an appendix:
\begin{align}
    \frac{\textup{d}}{\textup{d}t}\widehat{\mathbf{q}}(t)
    &= \mathbf{F}(\widehat{\mathbf{q}}(t), \mathbf{u}(t)).
\end{align}
To change how this looks, play with the \verb"\numberwithin{}{}" command in \texttt{config.tex} \textbf{and} the \verb"\numberwithin{}{}" command in \texttt{utdiss.sty} under \verb"\def\appendices".

\item The style of theorems, remarks, definitions, and so on is dictated by \verb"\theoremstyle{}" in \texttt{config.tex}.
If you change the theorem style and have appendices, you may also have to change the \verb"\theoremstyle{}`" in \texttt{utdiss.sty} under \verb"\def\appendices".
\end{itemize}

\newpage
\begin{figure}[t]
    \centering
    \layout*
    \vspace{.75in}
    \caption{Layout of the page setup.}
    \label{figure:layout}
\end{figure}
