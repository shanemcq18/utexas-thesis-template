\chapter{Using This Template}

This chapter discusses the anatomy of \texttt{main.tex}, the file that is compiled to produce the actual report.

\section{Preamble} % ==========================================================

The \emph{preamble} of a \texttt{.tex} file is everything that comes before \verb+\begin{document}+.
\index{preamble}

\subsection{Configuration} % --------------------------------------------------

The document \texttt{main.tex} starts with the following lines.
\begin{verbatim}
    \documentclass[12pt]{report}
    \usepackage{utdiss}
    % Imports ---------------------------------------------------------------------

% Required packages.
\usepackage{amsfonts}           % Write mathematics.
\usepackage{amscd}              % Write mathematics.
\usepackage{amsmath}            % Write mathematics.
\usepackage{amsthm}             % Write mathematics.
\usepackage{cleveref}           % Make references with \cref{}.
\usepackage{graphicx}           % Include figures from external files.
\usepackage{lmodern}            % Handle font sizes.
\usepackage{makeidx}            % For making an index.

% Optional packages.
% \usepackage{draftwatermark}   % "DRAFT" watermark in background.

% Packages used solely for the explanation in the template.
\usepackage{blindtext}          % Nonsense text.
\usepackage{layout}             % Draw the page layout with \layout*.
\usepackage{url}                % Typesetting URLs.
\usepackage{verbatim}           % Quote block of text verbatim.

% Page setup ------------------------------------------------------------------

%% Line spacing (uncomment one).
% \singlespacing \singlespacequote
% \oneandonehalfspacing \oneandonehalfspacequote        % DEFAULT
% \doublespacing \doublespacequote

%% If there are 10 or more sections, 10 or more subsections for a section,
%% etc., you need to adjust the Table of Contents with \longtocentry.
% \longtocentry

% Math ------------------------------------------------------------------------

\theoremstyle{plain}
\newtheorem{theorem}{Theorem}[chapter]
\newtheorem{corollary}[theorem]{Corollary}
\newtheorem{lemma}[theorem]{Lemma}
\newtheorem{proposition}[theorem]{Proposition}

\theoremstyle{definition}
\newtheorem{definition}{Definition}[chapter]

\theoremstyle{remark}
\newtheorem{remark}{Remark}[chapter]
\newtheorem*{notation}{Notation}

% Number equations and theorems by chapter (Theorem 2.1, 2.2, ...).
% \numberwithin{theorem}{chapter}
\numberwithin{equation}{chapter}
\crefformat{equation}{(#2#1#3)}

% Other customizations --------------------------------------------------------

\end{verbatim}

The first line, \verb+\documentclass[12pt]{report}+, declares \texttt{report} as the document class with an option of 12pt for the character size (which is slightly greater that the default 10pt, but it's what the Office of the Graduate School recommends).
You may include other options, as in any other \LaTeX{} document.

The second line, \verb+\usepackage{utdiss}+, loads the \texttt{utdiss} package, which contains a set of commands intended to produce a document fulfilling the official requirements for a doctoral dissertation or master's thesis or report.

The third line, \verb+% Imports ---------------------------------------------------------------------

% Required packages.
\usepackage{amsfonts}           % Write mathematics.
\usepackage{amscd}              % Write mathematics.
\usepackage{amsmath}            % Write mathematics.
\usepackage{amsthm}             % Write mathematics.
\usepackage{cleveref}           % Make references with \cref{}.
\usepackage{graphicx}           % Include figures from external files.
\usepackage{lmodern}            % Handle font sizes.
\usepackage{makeidx}            % For making an index.

% Optional packages.
% \usepackage{draftwatermark}   % "DRAFT" watermark in background.

% Packages used solely for the explanation in the template.
\usepackage{blindtext}          % Nonsense text.
\usepackage{layout}             % Draw the page layout with \layout*.
\usepackage{url}                % Typesetting URLs.
\usepackage{verbatim}           % Quote block of text verbatim.

% Page setup ------------------------------------------------------------------

%% Line spacing (uncomment one).
% \singlespacing \singlespacequote
% \oneandonehalfspacing \oneandonehalfspacequote        % DEFAULT
% \doublespacing \doublespacequote

%% If there are 10 or more sections, 10 or more subsections for a section,
%% etc., you need to adjust the Table of Contents with \longtocentry.
% \longtocentry

% Math ------------------------------------------------------------------------

\theoremstyle{plain}
\newtheorem{theorem}{Theorem}[chapter]
\newtheorem{corollary}[theorem]{Corollary}
\newtheorem{lemma}[theorem]{Lemma}
\newtheorem{proposition}[theorem]{Proposition}

\theoremstyle{definition}
\newtheorem{definition}{Definition}[chapter]

\theoremstyle{remark}
\newtheorem{remark}{Remark}[chapter]
\newtheorem*{notation}{Notation}

% Number equations and theorems by chapter (Theorem 2.1, 2.2, ...).
% \numberwithin{theorem}{chapter}
\numberwithin{equation}{chapter}
\crefformat{equation}{(#2#1#3)}

% Other customizations --------------------------------------------------------
+, executes \texttt{config.tex}, which contains package imports and other customizations.
This is the file that you should put any customizations like \verb+\newcommand{}+ macros.
Note that \texttt{config.tex} is where the line spacing is set, chosen by uncommenting \textbf{one} of the following lines.
\begin{verbatim}
    % \singlespacing \singlespacequote
    % \oneandonehalfspacing \oneandonehalfspacequote
    % \doublespacing \doublespacequote
\end{verbatim}
\index{commands!singlespacing@\verb+\singlespacing+}%
\index{commands!singlespacequote@\verb+\singlespacequote+}%
\index{commands!oneandonehalfspacing@\verb+\oneandonehalfspacing+}%
\index{commands!oneandonehalfspacequote@\verb+\oneandonehalfspacequote+}%
\index{commands!doublespacing@\verb+\doublespacing+}%
\index{commands!doublespacequote@\verb+\doublespacequote+}%
At the time of writing, the Graduate School recommends at least one-and-a-half spacing.

\subsection{Author Data} % -----------------------------------------------------

The remainder of the preamble gathers data about the author and the committee.
\begin{verbatim}
    \author{First Middle Last}
    \address{youremail@utexas.edu}
    \title{Title of Your Dissertation or Thesis}

    \supervisor{Supervisor Name}
    \committeemembers
        [Committee Member B]
        [Committee Member C]
        [Committee Member D]
        {Committee Member E}        % Note the curly braces!

    \previousdegrees{B.S., M.S.}
    \graduationmonth{May}
    \graduationyear{2023}
    \typist{the author}
\end{verbatim}
Most of these commands are self explanatory.
Some details:

\begin{itemize}
    \item \verb+\author{}+: \index{commands!author@\verb+\author{}+}%
    Your full, official University name.
    Mixed case (e.g., \texttt{McCluskey}) is fine.

    \item \verb+\address{}+: \index{commands!address@\verb+\address{}+}%
    An email address \textbf{that you will still be able to use after graduation}.
    The Graduate School recommends \textbf{not} using a physical address for privacy/security reasons.

    \item \verb+\title{}+: \index{commands!title@\verb+\title{}+}%
    Your dissertation title.
    If the title consists of more than one line, it should be in inverted pyramid form.
    You may have to specify the line breakings by inserting newlines with \verb+\\+ or \verb+\newline+.

    \item \verb+\supervisor{}+: \index{commands!supervisor@\verb+\supervisor{}+}%
    The name of your advisor, i.e., the chair of your committee.
    If you have co-supervisors, use \verb+\supervisor[First Supervisor]{Second Supervisor}+.

    \item \verb+\committeemembers{}+: \index{commands!committee@\verb+\committeemembers[]{}+}%
    The names of the members of your committee, in the order that you want them to appear.
    Each name is surrounded in brackets \verb"[]" except for the final name, which goes in curly braces \verb"{}".

    \item \verb+\previousdegrees{}+: \index{commands!previousdegrees@\verb+\previousdegrees{}+}%
    Your previous degree(s), i.e., \texttt{B.S.} or \texttt{B.S., M.S.} or similar.

    \item \verb+\graduationmonth{}+: \index{commands!graduationmonth@\verb+\graduationmonth{}+}%
    The month of your graduation (May, August, or December).
    Do not abbreviate the month.

    \item \verb+\graduationyear{}+: \index{commands!graduationyear@\verb+\graduationyear{}+}%
    The year of your graduation.
    Use a 4 digit number (e.g., \texttt{2023}).

    \item \verb+\typist{}+: \index{commands!typist@\verb+\typist{}+}%
    The person who typed this report.
    If you do it yourself, put ``\texttt{the author}''.
\end{itemize}

The preamble concludes with \verb+\makeindex+.
Leave that command there, even if you do not have an index as described in \Cref{sec:usage:index}

\subsection{Modifications for Master's Degrees} % -----------------------------

Reports and theses for master's degrees must also include \verb+\degree{}+ and \verb+\degreeabbr{}+ commands and EITHER \verb+\masterthesis+ OR \verb+\masterreport+ before \verb+\makeindex+.
\begin{verbatim}
    \degree{MASTER OF SCIENCE}          % Or MASTER OF ARTS, for example.
    \degreeabbr{M.S.}                   % Or M.A., for example.
    % \masterthesis                     % Uncomment ONE of these.
    % \masterreport                     % Uncomment ONE of these.
\end{verbatim}
\index{master's degree!thesis}
\index{master's degree!report}
\index{commands!degree@\verb+\degree{}+}
\index{commands!degreeabbr@\verb+\degreeabbr{}+}
\index{commands!masterthesis@\verb+\masterthesis+}
\index{commands!masterreport@\verb+\masterreport+}
By default the document is formated as a doctoral dissertation.

In addition, the \verb+\titlepage+ command must come before the \verb+\commcertpage+ command (see next section).

\section{Front Matter} % =======================================================

Immediately after \verb+\begin{document}+ are the following lines.
\begin{verbatim}
    \copyrightpage
    \commcertpage
    \titlepage

    \begin{dedication}
    A short dedication to someone special.
    \end{dedication}

    \begin{acknowledgments}
    Acknowledgments are technically optional, but come on, it takes a village. Say thanks!
\index{Acknowledgments@\emph{Acknowledgments}}
This section is not limited to a single page.

This \LaTeX{} template was originally written in 1991 by Young~U.~Ryu and later modified by Miguel~A.~Lerma and Craig~McCluskey.
Thanks you guys, we all owe you.
The template was heavily modified by Shane~A.~McQuarrie in 2023.
\index{template!history of}

    \end{acknowledgments}

    \utabstract
    \indent
    The abstract is a concise statement of the nature and content of your ETD, indicating its significance as a piece of research.
\index{Abstract@\emph{Abstract}}
It should be continuous prose, not disconnected notes or an outline.
It should be 1.5 or double spaced, not single spaced.
The abstract does not need to fit on a single page.

This document deals with how to write a doctoral dissertation using \LaTeX{} and the \texttt{utdiss} package.
Despite efforts to reconcile this template to the requirements set by the Graduate School, there is \textbf{no guarantee} that a dissertation generated by this template will fulfill all such requirements.
\index{disclaimer}
Guidelines, tips, and other information listed in this document are not officially sanctioned by the Graduate School, and the author remains responsible for verifying the correctness of the final work.
The latest version of this template can be found on GitHub at \url{https://github.com/shanemcq18/utexas-thesis-template}.
\index{GitHub}
\index{template!latest version}

Please note that this document is not a \LaTeX{} tutorial.
If you are just getting started with \LaTeX{}, we recommend \url{https://www.overleaf.com/learn/latex/Tutorials}.


    \tableofcontents
    \listoftables
    \listoffigures
\end{verbatim}
\index{commands!copyrightpage@\verb+\copyrightpage+}
\index{commands!commcertpage@\verb+\commcertpage+}
\index{commands!titlepage@\verb+\titlepage+}
\index{environments!dedication@\verb+dedication+}
\index{environments!acknowledgments@\verb+acknowledgments+}
\index{commands!utabstract@\verb+\utabstract+}
\index{commands!tableofcontents@\verb+\tableofcontents+}
\index{commands!listoftables@\verb+\listoftables+}
\index{commands!listoffigures@\verb+\listoffigures+}
Leave these lines alone except for the dedication, the lines between \verb+\begin{dedication}+ and \verb+\end{dedication}+.
Write something short, sweet, and possibly heartwarming within the \verb+\emph{}+.

The text of the abstract should be placed in \texttt{abstract.tex}.
Similarly, the text of the acknowledgments should be placed in \texttt{acknowledgments.tex}.

If there are 10 or more sections, 10 or more subsections for a section,
etc., you need adjust to the Table of Contents by using the
command \verb+\longtocentry+.
\index{commands!longtocentry@\verb+\longtocentry+}%
This command allocates the proper horizontal space for double-digit
numbers.

\section{Body} % ==============================================================

We've finally arrived at the actual text of the report, which should be organized into chapters and appendices.
If you are writing a short dissertation
that does not require chapters, use the command \verb+\nochapters+
\index{commands!nochapters@\verb+\nochapters+}%
Otherwise, we strongly recommend\footnote{See \url{https://www.overleaf.com/learn/latex/Management_in_a_large_project}.} placing the source text for each chapter or appendix in its own files to allow chapters to be easily inserted, re-ordered, or removed.
This way, the next chunk of \texttt{main.tex} will then looks something like this:
\begin{verbatim}
    \include{chapter-introduction}
    \chapter{Using This Template}

This chapter discusses the anatomy of \texttt{main.tex}, the file that is compiled to produce the actual report.

\section{Preamble} % ==========================================================

The \emph{preamble} of a \texttt{.tex} file is everything that comes before \verb+\begin{document}+.
\index{preamble}

\subsection{Configuration} % --------------------------------------------------

The document \texttt{main.tex} starts with the following lines.
\begin{verbatim}
    \documentclass[12pt]{report}
    \usepackage{utdiss}
    % Imports ---------------------------------------------------------------------

% Required packages.
\usepackage{amsfonts}           % Write mathematics.
\usepackage{amscd}              % Write mathematics.
\usepackage{amsmath}            % Write mathematics.
\usepackage{amsthm}             % Write mathematics.
\usepackage{cleveref}           % Make references with \cref{}.
\usepackage{graphicx}           % Include figures from external files.
\usepackage{lmodern}            % Handle font sizes.
\usepackage{makeidx}            % For making an index.

% Optional packages.
% \usepackage{draftwatermark}   % "DRAFT" watermark in background.

% Packages used solely for the explanation in the template.
\usepackage{blindtext}          % Nonsense text.
\usepackage{layout}             % Draw the page layout with \layout*.
\usepackage{url}                % Typesetting URLs.
\usepackage{verbatim}           % Quote block of text verbatim.

% Page setup ------------------------------------------------------------------

%% Line spacing (uncomment one).
% \singlespacing \singlespacequote
% \oneandonehalfspacing \oneandonehalfspacequote        % DEFAULT
% \doublespacing \doublespacequote

%% If there are 10 or more sections, 10 or more subsections for a section,
%% etc., you need to adjust the Table of Contents with \longtocentry.
% \longtocentry

% Math ------------------------------------------------------------------------

\theoremstyle{plain}
\newtheorem{theorem}{Theorem}[chapter]
\newtheorem{corollary}[theorem]{Corollary}
\newtheorem{lemma}[theorem]{Lemma}
\newtheorem{proposition}[theorem]{Proposition}

\theoremstyle{definition}
\newtheorem{definition}{Definition}[chapter]

\theoremstyle{remark}
\newtheorem{remark}{Remark}[chapter]
\newtheorem*{notation}{Notation}

% Number equations and theorems by chapter (Theorem 2.1, 2.2, ...).
% \numberwithin{theorem}{chapter}
\numberwithin{equation}{chapter}
\crefformat{equation}{(#2#1#3)}

% Other customizations --------------------------------------------------------

\end{verbatim}

The first line, \verb+\documentclass[12pt]{report}+, declares \texttt{report} as the document class with an option of 12pt for the character size (which is slightly greater that the default 10pt, but it's what the Office of the Graduate School recommends).
You may include other options, as in any other \LaTeX{} document.

The second line, \verb+\usepackage{utdiss}+, loads the \texttt{utdiss} package, which contains a set of commands intended to produce a document fulfilling the official requirements for a doctoral dissertation or master's thesis or report.

The third line, \verb+% Imports ---------------------------------------------------------------------

% Required packages.
\usepackage{amsfonts}           % Write mathematics.
\usepackage{amscd}              % Write mathematics.
\usepackage{amsmath}            % Write mathematics.
\usepackage{amsthm}             % Write mathematics.
\usepackage{cleveref}           % Make references with \cref{}.
\usepackage{graphicx}           % Include figures from external files.
\usepackage{lmodern}            % Handle font sizes.
\usepackage{makeidx}            % For making an index.

% Optional packages.
% \usepackage{draftwatermark}   % "DRAFT" watermark in background.

% Packages used solely for the explanation in the template.
\usepackage{blindtext}          % Nonsense text.
\usepackage{layout}             % Draw the page layout with \layout*.
\usepackage{url}                % Typesetting URLs.
\usepackage{verbatim}           % Quote block of text verbatim.

% Page setup ------------------------------------------------------------------

%% Line spacing (uncomment one).
% \singlespacing \singlespacequote
% \oneandonehalfspacing \oneandonehalfspacequote        % DEFAULT
% \doublespacing \doublespacequote

%% If there are 10 or more sections, 10 or more subsections for a section,
%% etc., you need to adjust the Table of Contents with \longtocentry.
% \longtocentry

% Math ------------------------------------------------------------------------

\theoremstyle{plain}
\newtheorem{theorem}{Theorem}[chapter]
\newtheorem{corollary}[theorem]{Corollary}
\newtheorem{lemma}[theorem]{Lemma}
\newtheorem{proposition}[theorem]{Proposition}

\theoremstyle{definition}
\newtheorem{definition}{Definition}[chapter]

\theoremstyle{remark}
\newtheorem{remark}{Remark}[chapter]
\newtheorem*{notation}{Notation}

% Number equations and theorems by chapter (Theorem 2.1, 2.2, ...).
% \numberwithin{theorem}{chapter}
\numberwithin{equation}{chapter}
\crefformat{equation}{(#2#1#3)}

% Other customizations --------------------------------------------------------
+, executes \texttt{config.tex}, which contains package imports and other customizations.
This is the file that you should put any customizations like \verb+\newcommand{}+ macros.
Note that \texttt{config.tex} is where the line spacing is set, chosen by uncommenting \textbf{one} of the following lines.
\begin{verbatim}
    % \singlespacing \singlespacequote
    % \oneandonehalfspacing \oneandonehalfspacequote
    % \doublespacing \doublespacequote
\end{verbatim}
\index{commands!singlespacing@\verb+\singlespacing+}%
\index{commands!singlespacequote@\verb+\singlespacequote+}%
\index{commands!oneandonehalfspacing@\verb+\oneandonehalfspacing+}%
\index{commands!oneandonehalfspacequote@\verb+\oneandonehalfspacequote+}%
\index{commands!doublespacing@\verb+\doublespacing+}%
\index{commands!doublespacequote@\verb+\doublespacequote+}%
At the time of writing, the Graduate School recommends at least one-and-a-half spacing.

\subsection{Author Data} % -----------------------------------------------------

The remainder of the preamble gathers data about the author and the committee.
\begin{verbatim}
    \author{First Middle Last}
    \address{youremail@utexas.edu}
    \title{Title of Your Dissertation or Thesis}

    \supervisor{Supervisor Name}
    \committeemembers
        [Committee Member B]
        [Committee Member C]
        [Committee Member D]
        {Committee Member E}        % Note the curly braces!

    \previousdegrees{B.S., M.S.}
    \graduationmonth{May}
    \graduationyear{2023}
    \typist{the author}
\end{verbatim}
Most of these commands are self explanatory.
Some details:

\begin{itemize}
    \item \verb+\author{}+: \index{commands!author@\verb+\author{}+}%
    Your full, official University name.
    Mixed case (e.g., \texttt{McCluskey}) is fine.

    \item \verb+\address{}+: \index{commands!address@\verb+\address{}+}%
    An email address \textbf{that you will still be able to use after graduation}.
    The Graduate School recommends \textbf{not} using a physical address for privacy/security reasons.

    \item \verb+\title{}+: \index{commands!title@\verb+\title{}+}%
    Your dissertation title.
    If the title consists of more than one line, it should be in inverted pyramid form.
    You may have to specify the line breakings by inserting newlines with \verb+\\+ or \verb+\newline+.

    \item \verb+\supervisor{}+: \index{commands!supervisor@\verb+\supervisor{}+}%
    The name of your advisor, i.e., the chair of your committee.
    If you have co-supervisors, use \verb+\supervisor[First Supervisor]{Second Supervisor}+.

    \item \verb+\committeemembers{}+: \index{commands!committee@\verb+\committeemembers[]{}+}%
    The names of the members of your committee, in the order that you want them to appear.
    Each name is surrounded in brackets \verb"[]" except for the final name, which goes in curly braces \verb"{}".

    \item \verb+\previousdegrees{}+: \index{commands!previousdegrees@\verb+\previousdegrees{}+}%
    Your previous degree(s), i.e., \texttt{B.S.} or \texttt{B.S., M.S.} or similar.

    \item \verb+\graduationmonth{}+: \index{commands!graduationmonth@\verb+\graduationmonth{}+}%
    The month of your graduation (May, August, or December).
    Do not abbreviate the month.

    \item \verb+\graduationyear{}+: \index{commands!graduationyear@\verb+\graduationyear{}+}%
    The year of your graduation.
    Use a 4 digit number (e.g., \texttt{2023}).

    \item \verb+\typist{}+: \index{commands!typist@\verb+\typist{}+}%
    The person who typed this report.
    If you do it yourself, put ``\texttt{the author}''.
\end{itemize}

The preamble concludes with \verb+\makeindex+.
Leave that command there, even if you do not have an index as described in \Cref{sec:usage:index}

\subsection{Modifications for Master's Degrees} % -----------------------------

Reports and theses for master's degrees must also include \verb+\degree{}+ and \verb+\degreeabbr{}+ commands and EITHER \verb+\masterthesis+ OR \verb+\masterreport+ before \verb+\makeindex+.
\begin{verbatim}
    \degree{MASTER OF SCIENCE}          % Or MASTER OF ARTS, for example.
    \degreeabbr{M.S.}                   % Or M.A., for example.
    % \masterthesis                     % Uncomment ONE of these.
    % \masterreport                     % Uncomment ONE of these.
\end{verbatim}
\index{master's degree!thesis}
\index{master's degree!report}
\index{commands!degree@\verb+\degree{}+}
\index{commands!degreeabbr@\verb+\degreeabbr{}+}
\index{commands!masterthesis@\verb+\masterthesis+}
\index{commands!masterreport@\verb+\masterreport+}
By default the document is formated as a doctoral dissertation.

In addition, the \verb+\titlepage+ command must come before the \verb+\commcertpage+ command (see next section).

\section{Front Matter} % =======================================================

Immediately after \verb+\begin{document}+ are the following lines.
\begin{verbatim}
    \copyrightpage
    \commcertpage
    \titlepage

    \begin{dedication}
    A short dedication to someone special.
    \end{dedication}

    \begin{acknowledgments}
    Acknowledgments are technically optional, but come on, it takes a village. Say thanks!
\index{Acknowledgments@\emph{Acknowledgments}}
This section is not limited to a single page.

This \LaTeX{} template was originally written in 1991 by Young~U.~Ryu and later modified by Miguel~A.~Lerma and Craig~McCluskey.
Thanks you guys, we all owe you.
The template was heavily modified by Shane~A.~McQuarrie in 2023.
\index{template!history of}

    \end{acknowledgments}

    \utabstract
    \indent
    The abstract is a concise statement of the nature and content of your ETD, indicating its significance as a piece of research.
\index{Abstract@\emph{Abstract}}
It should be continuous prose, not disconnected notes or an outline.
It should be 1.5 or double spaced, not single spaced.
The abstract does not need to fit on a single page.

This document deals with how to write a doctoral dissertation using \LaTeX{} and the \texttt{utdiss} package.
Despite efforts to reconcile this template to the requirements set by the Graduate School, there is \textbf{no guarantee} that a dissertation generated by this template will fulfill all such requirements.
\index{disclaimer}
Guidelines, tips, and other information listed in this document are not officially sanctioned by the Graduate School, and the author remains responsible for verifying the correctness of the final work.
The latest version of this template can be found on GitHub at \url{https://github.com/shanemcq18/utexas-thesis-template}.
\index{GitHub}
\index{template!latest version}

Please note that this document is not a \LaTeX{} tutorial.
If you are just getting started with \LaTeX{}, we recommend \url{https://www.overleaf.com/learn/latex/Tutorials}.


    \tableofcontents
    \listoftables
    \listoffigures
\end{verbatim}
\index{commands!copyrightpage@\verb+\copyrightpage+}
\index{commands!commcertpage@\verb+\commcertpage+}
\index{commands!titlepage@\verb+\titlepage+}
\index{environments!dedication@\verb+dedication+}
\index{environments!acknowledgments@\verb+acknowledgments+}
\index{commands!utabstract@\verb+\utabstract+}
\index{commands!tableofcontents@\verb+\tableofcontents+}
\index{commands!listoftables@\verb+\listoftables+}
\index{commands!listoffigures@\verb+\listoffigures+}
Leave these lines alone except for the dedication, the lines between \verb+\begin{dedication}+ and \verb+\end{dedication}+.
Write something short, sweet, and possibly heartwarming within the \verb+\emph{}+.

The text of the abstract should be placed in \texttt{abstract.tex}.
Similarly, the text of the acknowledgments should be placed in \texttt{acknowledgments.tex}.

If there are 10 or more sections, 10 or more subsections for a section,
etc., you need adjust to the Table of Contents by using the
command \verb+\longtocentry+.
\index{commands!longtocentry@\verb+\longtocentry+}%
This command allocates the proper horizontal space for double-digit
numbers.

\section{Body} % ==============================================================

We've finally arrived at the actual text of the report, which should be organized into chapters and appendices.
If you are writing a short dissertation
that does not require chapters, use the command \verb+\nochapters+
\index{commands!nochapters@\verb+\nochapters+}%
Otherwise, we strongly recommend\footnote{See \url{https://www.overleaf.com/learn/latex/Management_in_a_large_project}.} placing the source text for each chapter or appendix in its own files to allow chapters to be easily inserted, re-ordered, or removed.
This way, the next chunk of \texttt{main.tex} will then looks something like this:
\begin{verbatim}
    \include{chapter-introduction}
    \chapter{Using This Template}

This chapter discusses the anatomy of \texttt{main.tex}, the file that is compiled to produce the actual report.

\section{Preamble} % ==========================================================

The \emph{preamble} of a \texttt{.tex} file is everything that comes before \verb+\begin{document}+.
\index{preamble}

\subsection{Configuration} % --------------------------------------------------

The document \texttt{main.tex} starts with the following lines.
\begin{verbatim}
    \documentclass[12pt]{report}
    \usepackage{utdiss}
    % Imports ---------------------------------------------------------------------

% Required packages.
\usepackage{amsfonts}           % Write mathematics.
\usepackage{amscd}              % Write mathematics.
\usepackage{amsmath}            % Write mathematics.
\usepackage{amsthm}             % Write mathematics.
\usepackage{cleveref}           % Make references with \cref{}.
\usepackage{graphicx}           % Include figures from external files.
\usepackage{lmodern}            % Handle font sizes.
\usepackage{makeidx}            % For making an index.

% Optional packages.
% \usepackage{draftwatermark}   % "DRAFT" watermark in background.

% Packages used solely for the explanation in the template.
\usepackage{blindtext}          % Nonsense text.
\usepackage{layout}             % Draw the page layout with \layout*.
\usepackage{url}                % Typesetting URLs.
\usepackage{verbatim}           % Quote block of text verbatim.

% Page setup ------------------------------------------------------------------

%% Line spacing (uncomment one).
% \singlespacing \singlespacequote
% \oneandonehalfspacing \oneandonehalfspacequote        % DEFAULT
% \doublespacing \doublespacequote

%% If there are 10 or more sections, 10 or more subsections for a section,
%% etc., you need to adjust the Table of Contents with \longtocentry.
% \longtocentry

% Math ------------------------------------------------------------------------

\theoremstyle{plain}
\newtheorem{theorem}{Theorem}[chapter]
\newtheorem{corollary}[theorem]{Corollary}
\newtheorem{lemma}[theorem]{Lemma}
\newtheorem{proposition}[theorem]{Proposition}

\theoremstyle{definition}
\newtheorem{definition}{Definition}[chapter]

\theoremstyle{remark}
\newtheorem{remark}{Remark}[chapter]
\newtheorem*{notation}{Notation}

% Number equations and theorems by chapter (Theorem 2.1, 2.2, ...).
% \numberwithin{theorem}{chapter}
\numberwithin{equation}{chapter}
\crefformat{equation}{(#2#1#3)}

% Other customizations --------------------------------------------------------

\end{verbatim}

The first line, \verb+\documentclass[12pt]{report}+, declares \texttt{report} as the document class with an option of 12pt for the character size (which is slightly greater that the default 10pt, but it's what the Office of the Graduate School recommends).
You may include other options, as in any other \LaTeX{} document.

The second line, \verb+\usepackage{utdiss}+, loads the \texttt{utdiss} package, which contains a set of commands intended to produce a document fulfilling the official requirements for a doctoral dissertation or master's thesis or report.

The third line, \verb+% Imports ---------------------------------------------------------------------

% Required packages.
\usepackage{amsfonts}           % Write mathematics.
\usepackage{amscd}              % Write mathematics.
\usepackage{amsmath}            % Write mathematics.
\usepackage{amsthm}             % Write mathematics.
\usepackage{cleveref}           % Make references with \cref{}.
\usepackage{graphicx}           % Include figures from external files.
\usepackage{lmodern}            % Handle font sizes.
\usepackage{makeidx}            % For making an index.

% Optional packages.
% \usepackage{draftwatermark}   % "DRAFT" watermark in background.

% Packages used solely for the explanation in the template.
\usepackage{blindtext}          % Nonsense text.
\usepackage{layout}             % Draw the page layout with \layout*.
\usepackage{url}                % Typesetting URLs.
\usepackage{verbatim}           % Quote block of text verbatim.

% Page setup ------------------------------------------------------------------

%% Line spacing (uncomment one).
% \singlespacing \singlespacequote
% \oneandonehalfspacing \oneandonehalfspacequote        % DEFAULT
% \doublespacing \doublespacequote

%% If there are 10 or more sections, 10 or more subsections for a section,
%% etc., you need to adjust the Table of Contents with \longtocentry.
% \longtocentry

% Math ------------------------------------------------------------------------

\theoremstyle{plain}
\newtheorem{theorem}{Theorem}[chapter]
\newtheorem{corollary}[theorem]{Corollary}
\newtheorem{lemma}[theorem]{Lemma}
\newtheorem{proposition}[theorem]{Proposition}

\theoremstyle{definition}
\newtheorem{definition}{Definition}[chapter]

\theoremstyle{remark}
\newtheorem{remark}{Remark}[chapter]
\newtheorem*{notation}{Notation}

% Number equations and theorems by chapter (Theorem 2.1, 2.2, ...).
% \numberwithin{theorem}{chapter}
\numberwithin{equation}{chapter}
\crefformat{equation}{(#2#1#3)}

% Other customizations --------------------------------------------------------
+, executes \texttt{config.tex}, which contains package imports and other customizations.
This is the file that you should put any customizations like \verb+\newcommand{}+ macros.
Note that \texttt{config.tex} is where the line spacing is set, chosen by uncommenting \textbf{one} of the following lines.
\begin{verbatim}
    % \singlespacing \singlespacequote
    % \oneandonehalfspacing \oneandonehalfspacequote
    % \doublespacing \doublespacequote
\end{verbatim}
\index{commands!singlespacing@\verb+\singlespacing+}%
\index{commands!singlespacequote@\verb+\singlespacequote+}%
\index{commands!oneandonehalfspacing@\verb+\oneandonehalfspacing+}%
\index{commands!oneandonehalfspacequote@\verb+\oneandonehalfspacequote+}%
\index{commands!doublespacing@\verb+\doublespacing+}%
\index{commands!doublespacequote@\verb+\doublespacequote+}%
At the time of writing, the Graduate School recommends at least one-and-a-half spacing.

\subsection{Author Data} % -----------------------------------------------------

The remainder of the preamble gathers data about the author and the committee.
\begin{verbatim}
    \author{First Middle Last}
    \address{youremail@utexas.edu}
    \title{Title of Your Dissertation or Thesis}

    \supervisor{Supervisor Name}
    \committeemembers
        [Committee Member B]
        [Committee Member C]
        [Committee Member D]
        {Committee Member E}        % Note the curly braces!

    \previousdegrees{B.S., M.S.}
    \graduationmonth{May}
    \graduationyear{2023}
    \typist{the author}
\end{verbatim}
Most of these commands are self explanatory.
Some details:

\begin{itemize}
    \item \verb+\author{}+: \index{commands!author@\verb+\author{}+}%
    Your full, official University name.
    Mixed case (e.g., \texttt{McCluskey}) is fine.

    \item \verb+\address{}+: \index{commands!address@\verb+\address{}+}%
    An email address \textbf{that you will still be able to use after graduation}.
    The Graduate School recommends \textbf{not} using a physical address for privacy/security reasons.

    \item \verb+\title{}+: \index{commands!title@\verb+\title{}+}%
    Your dissertation title.
    If the title consists of more than one line, it should be in inverted pyramid form.
    You may have to specify the line breakings by inserting newlines with \verb+\\+ or \verb+\newline+.

    \item \verb+\supervisor{}+: \index{commands!supervisor@\verb+\supervisor{}+}%
    The name of your advisor, i.e., the chair of your committee.
    If you have co-supervisors, use \verb+\supervisor[First Supervisor]{Second Supervisor}+.

    \item \verb+\committeemembers{}+: \index{commands!committee@\verb+\committeemembers[]{}+}%
    The names of the members of your committee, in the order that you want them to appear.
    Each name is surrounded in brackets \verb"[]" except for the final name, which goes in curly braces \verb"{}".

    \item \verb+\previousdegrees{}+: \index{commands!previousdegrees@\verb+\previousdegrees{}+}%
    Your previous degree(s), i.e., \texttt{B.S.} or \texttt{B.S., M.S.} or similar.

    \item \verb+\graduationmonth{}+: \index{commands!graduationmonth@\verb+\graduationmonth{}+}%
    The month of your graduation (May, August, or December).
    Do not abbreviate the month.

    \item \verb+\graduationyear{}+: \index{commands!graduationyear@\verb+\graduationyear{}+}%
    The year of your graduation.
    Use a 4 digit number (e.g., \texttt{2023}).

    \item \verb+\typist{}+: \index{commands!typist@\verb+\typist{}+}%
    The person who typed this report.
    If you do it yourself, put ``\texttt{the author}''.
\end{itemize}

The preamble concludes with \verb+\makeindex+.
Leave that command there, even if you do not have an index as described in \Cref{sec:usage:index}

\subsection{Modifications for Master's Degrees} % -----------------------------

Reports and theses for master's degrees must also include \verb+\degree{}+ and \verb+\degreeabbr{}+ commands and EITHER \verb+\masterthesis+ OR \verb+\masterreport+ before \verb+\makeindex+.
\begin{verbatim}
    \degree{MASTER OF SCIENCE}          % Or MASTER OF ARTS, for example.
    \degreeabbr{M.S.}                   % Or M.A., for example.
    % \masterthesis                     % Uncomment ONE of these.
    % \masterreport                     % Uncomment ONE of these.
\end{verbatim}
\index{master's degree!thesis}
\index{master's degree!report}
\index{commands!degree@\verb+\degree{}+}
\index{commands!degreeabbr@\verb+\degreeabbr{}+}
\index{commands!masterthesis@\verb+\masterthesis+}
\index{commands!masterreport@\verb+\masterreport+}
By default the document is formated as a doctoral dissertation.

In addition, the \verb+\titlepage+ command must come before the \verb+\commcertpage+ command (see next section).

\section{Front Matter} % =======================================================

Immediately after \verb+\begin{document}+ are the following lines.
\begin{verbatim}
    \copyrightpage
    \commcertpage
    \titlepage

    \begin{dedication}
    A short dedication to someone special.
    \end{dedication}

    \begin{acknowledgments}
    Acknowledgments are technically optional, but come on, it takes a village. Say thanks!
\index{Acknowledgments@\emph{Acknowledgments}}
This section is not limited to a single page.

This \LaTeX{} template was originally written in 1991 by Young~U.~Ryu and later modified by Miguel~A.~Lerma and Craig~McCluskey.
Thanks you guys, we all owe you.
The template was heavily modified by Shane~A.~McQuarrie in 2023.
\index{template!history of}

    \end{acknowledgments}

    \utabstract
    \indent
    The abstract is a concise statement of the nature and content of your ETD, indicating its significance as a piece of research.
\index{Abstract@\emph{Abstract}}
It should be continuous prose, not disconnected notes or an outline.
It should be 1.5 or double spaced, not single spaced.
The abstract does not need to fit on a single page.

This document deals with how to write a doctoral dissertation using \LaTeX{} and the \texttt{utdiss} package.
Despite efforts to reconcile this template to the requirements set by the Graduate School, there is \textbf{no guarantee} that a dissertation generated by this template will fulfill all such requirements.
\index{disclaimer}
Guidelines, tips, and other information listed in this document are not officially sanctioned by the Graduate School, and the author remains responsible for verifying the correctness of the final work.
The latest version of this template can be found on GitHub at \url{https://github.com/shanemcq18/utexas-thesis-template}.
\index{GitHub}
\index{template!latest version}

Please note that this document is not a \LaTeX{} tutorial.
If you are just getting started with \LaTeX{}, we recommend \url{https://www.overleaf.com/learn/latex/Tutorials}.


    \tableofcontents
    \listoftables
    \listoffigures
\end{verbatim}
\index{commands!copyrightpage@\verb+\copyrightpage+}
\index{commands!commcertpage@\verb+\commcertpage+}
\index{commands!titlepage@\verb+\titlepage+}
\index{environments!dedication@\verb+dedication+}
\index{environments!acknowledgments@\verb+acknowledgments+}
\index{commands!utabstract@\verb+\utabstract+}
\index{commands!tableofcontents@\verb+\tableofcontents+}
\index{commands!listoftables@\verb+\listoftables+}
\index{commands!listoffigures@\verb+\listoffigures+}
Leave these lines alone except for the dedication, the lines between \verb+\begin{dedication}+ and \verb+\end{dedication}+.
Write something short, sweet, and possibly heartwarming within the \verb+\emph{}+.

The text of the abstract should be placed in \texttt{abstract.tex}.
Similarly, the text of the acknowledgments should be placed in \texttt{acknowledgments.tex}.

If there are 10 or more sections, 10 or more subsections for a section,
etc., you need adjust to the Table of Contents by using the
command \verb+\longtocentry+.
\index{commands!longtocentry@\verb+\longtocentry+}%
This command allocates the proper horizontal space for double-digit
numbers.

\section{Body} % ==============================================================

We've finally arrived at the actual text of the report, which should be organized into chapters and appendices.
If you are writing a short dissertation
that does not require chapters, use the command \verb+\nochapters+
\index{commands!nochapters@\verb+\nochapters+}%
Otherwise, we strongly recommend\footnote{See \url{https://www.overleaf.com/learn/latex/Management_in_a_large_project}.} placing the source text for each chapter or appendix in its own files to allow chapters to be easily inserted, re-ordered, or removed.
This way, the next chunk of \texttt{main.tex} will then looks something like this:
\begin{verbatim}
    \include{chapter-introduction}
    \chapter{Using This Template}

This chapter discusses the anatomy of \texttt{main.tex}, the file that is compiled to produce the actual report.

\section{Preamble} % ==========================================================

The \emph{preamble} of a \texttt{.tex} file is everything that comes before \verb+\begin{document}+.
\index{preamble}

\subsection{Configuration} % --------------------------------------------------

The document \texttt{main.tex} starts with the following lines.
\begin{verbatim}
    \documentclass[12pt]{report}
    \usepackage{utdiss}
    \input{config}
\end{verbatim}

The first line, \verb+\documentclass[12pt]{report}+, declares \texttt{report} as the document class with an option of 12pt for the character size (which is slightly greater that the default 10pt, but it's what the Office of the Graduate School recommends).
You may include other options, as in any other \LaTeX{} document.

The second line, \verb+\usepackage{utdiss}+, loads the \texttt{utdiss} package, which contains a set of commands intended to produce a document fulfilling the official requirements for a doctoral dissertation or master's thesis or report.

The third line, \verb+\input{config}+, executes \texttt{config.tex}, which contains package imports and other customizations.
This is the file that you should put any customizations like \verb+\newcommand{}+ macros.
Note that \texttt{config.tex} is where the line spacing is set, chosen by uncommenting \textbf{one} of the following lines.
\begin{verbatim}
    % \singlespacing \singlespacequote
    % \oneandonehalfspacing \oneandonehalfspacequote
    % \doublespacing \doublespacequote
\end{verbatim}
\index{commands!singlespacing@\verb+\singlespacing+}%
\index{commands!singlespacequote@\verb+\singlespacequote+}%
\index{commands!oneandonehalfspacing@\verb+\oneandonehalfspacing+}%
\index{commands!oneandonehalfspacequote@\verb+\oneandonehalfspacequote+}%
\index{commands!doublespacing@\verb+\doublespacing+}%
\index{commands!doublespacequote@\verb+\doublespacequote+}%
At the time of writing, the Graduate School recommends at least one-and-a-half spacing.

\subsection{Author Data} % -----------------------------------------------------

The remainder of the preamble gathers data about the author and the committee.
\begin{verbatim}
    \author{First Middle Last}
    \address{youremail@utexas.edu}
    \title{Title of Your Dissertation or Thesis}

    \supervisor{Supervisor Name}
    \committeemembers
        [Committee Member B]
        [Committee Member C]
        [Committee Member D]
        {Committee Member E}        % Note the curly braces!

    \previousdegrees{B.S., M.S.}
    \graduationmonth{May}
    \graduationyear{2023}
    \typist{the author}
\end{verbatim}
Most of these commands are self explanatory.
Some details:

\begin{itemize}
    \item \verb+\author{}+: \index{commands!author@\verb+\author{}+}%
    Your full, official University name.
    Mixed case (e.g., \texttt{McCluskey}) is fine.

    \item \verb+\address{}+: \index{commands!address@\verb+\address{}+}%
    An email address \textbf{that you will still be able to use after graduation}.
    The Graduate School recommends \textbf{not} using a physical address for privacy/security reasons.

    \item \verb+\title{}+: \index{commands!title@\verb+\title{}+}%
    Your dissertation title.
    If the title consists of more than one line, it should be in inverted pyramid form.
    You may have to specify the line breakings by inserting newlines with \verb+\\+ or \verb+\newline+.

    \item \verb+\supervisor{}+: \index{commands!supervisor@\verb+\supervisor{}+}%
    The name of your advisor, i.e., the chair of your committee.
    If you have co-supervisors, use \verb+\supervisor[First Supervisor]{Second Supervisor}+.

    \item \verb+\committeemembers{}+: \index{commands!committee@\verb+\committeemembers[]{}+}%
    The names of the members of your committee, in the order that you want them to appear.
    Each name is surrounded in brackets \verb"[]" except for the final name, which goes in curly braces \verb"{}".

    \item \verb+\previousdegrees{}+: \index{commands!previousdegrees@\verb+\previousdegrees{}+}%
    Your previous degree(s), i.e., \texttt{B.S.} or \texttt{B.S., M.S.} or similar.

    \item \verb+\graduationmonth{}+: \index{commands!graduationmonth@\verb+\graduationmonth{}+}%
    The month of your graduation (May, August, or December).
    Do not abbreviate the month.

    \item \verb+\graduationyear{}+: \index{commands!graduationyear@\verb+\graduationyear{}+}%
    The year of your graduation.
    Use a 4 digit number (e.g., \texttt{2023}).

    \item \verb+\typist{}+: \index{commands!typist@\verb+\typist{}+}%
    The person who typed this report.
    If you do it yourself, put ``\texttt{the author}''.
\end{itemize}

The preamble concludes with \verb+\makeindex+.
Leave that command there, even if you do not have an index as described in \Cref{sec:usage:index}

\subsection{Modifications for Master's Degrees} % -----------------------------

Reports and theses for master's degrees must also include \verb+\degree{}+ and \verb+\degreeabbr{}+ commands and EITHER \verb+\masterthesis+ OR \verb+\masterreport+ before \verb+\makeindex+.
\begin{verbatim}
    \degree{MASTER OF SCIENCE}          % Or MASTER OF ARTS, for example.
    \degreeabbr{M.S.}                   % Or M.A., for example.
    % \masterthesis                     % Uncomment ONE of these.
    % \masterreport                     % Uncomment ONE of these.
\end{verbatim}
\index{master's degree!thesis}
\index{master's degree!report}
\index{commands!degree@\verb+\degree{}+}
\index{commands!degreeabbr@\verb+\degreeabbr{}+}
\index{commands!masterthesis@\verb+\masterthesis+}
\index{commands!masterreport@\verb+\masterreport+}
By default the document is formated as a doctoral dissertation.

In addition, the \verb+\titlepage+ command must come before the \verb+\commcertpage+ command (see next section).

\section{Front Matter} % =======================================================

Immediately after \verb+\begin{document}+ are the following lines.
\begin{verbatim}
    \copyrightpage
    \commcertpage
    \titlepage

    \begin{dedication}
    A short dedication to someone special.
    \end{dedication}

    \begin{acknowledgments}
    \input{acknowledgments}
    \end{acknowledgments}

    \utabstract
    \indent
    \input{abstract}

    \tableofcontents
    \listoftables
    \listoffigures
\end{verbatim}
\index{commands!copyrightpage@\verb+\copyrightpage+}
\index{commands!commcertpage@\verb+\commcertpage+}
\index{commands!titlepage@\verb+\titlepage+}
\index{environments!dedication@\verb+dedication+}
\index{environments!acknowledgments@\verb+acknowledgments+}
\index{commands!utabstract@\verb+\utabstract+}
\index{commands!tableofcontents@\verb+\tableofcontents+}
\index{commands!listoftables@\verb+\listoftables+}
\index{commands!listoffigures@\verb+\listoffigures+}
Leave these lines alone except for the dedication, the lines between \verb+\begin{dedication}+ and \verb+\end{dedication}+.
Write something short, sweet, and possibly heartwarming within the \verb+\emph{}+.

The text of the abstract should be placed in \texttt{abstract.tex}.
Similarly, the text of the acknowledgments should be placed in \texttt{acknowledgments.tex}.

If there are 10 or more sections, 10 or more subsections for a section,
etc., you need adjust to the Table of Contents by using the
command \verb+\longtocentry+.
\index{commands!longtocentry@\verb+\longtocentry+}%
This command allocates the proper horizontal space for double-digit
numbers.

\section{Body} % ==============================================================

We've finally arrived at the actual text of the report, which should be organized into chapters and appendices.
If you are writing a short dissertation
that does not require chapters, use the command \verb+\nochapters+
\index{commands!nochapters@\verb+\nochapters+}%
Otherwise, we strongly recommend\footnote{See \url{https://www.overleaf.com/learn/latex/Management_in_a_large_project}.} placing the source text for each chapter or appendix in its own files to allow chapters to be easily inserted, re-ordered, or removed.
This way, the next chunk of \texttt{main.tex} will then looks something like this:
\begin{verbatim}
    \include{chapter-introduction}
    \include{chapter-usage}
    \include{chapter-examples}
    \include{chapter-conclusion}
\end{verbatim}
\index{commands!include@\verb+\include{}+}%

\subsection{Chapters} % -------------------------------------------------------

The beginning of each chapter file should begin with a \verb+\chapter[]{}+
\index{commands!chapter@\verb+\chapter{}+}%
command, which is similar to \verb"section{}".
The part in \verb"[]" will be listed in the table of contents; the part in \verb"{}" will be printed in the body of the document.
Use \verb+\chapter{}+ (without \verb"[]") to use the same title for both the table of contents and the body.

If your chapter heading consists of more than one line, it will be automatically broken into separate lines.
If you don't like the way LaTeX breaks the chapter heading into lines, use \verb+\newheadline+ to break lines.
\textbf{Never use} \verb+\\+ \textbf{in sectional headings} (\verb+\chapter{}+, \verb+\section{}+, \verb+\subsection{}+, etc.)!
\index{commands!chapter@\verb+\chapter{}+}%
\index{commands!section@\verb+\section{}+}%
\index{commands!subsection@\verb+\subsection{}+}%

\subsection{Appendices} % -----------------------------------------------------
\index{Appendices@\emph{Appendices}}%

If you have no appendices, then proceed to the end matter.

If you have a single appendix, use \verb+\appendix+ and \textbf{do not use}
\verb+\chapter{}+ at the start of the appendix text.
\index{commands!appendix@\verb+\appendix+}
Otherwise the template will insert an extra page with only the word
`Appendix' on it and cause the actual appendix to be labeled `Appendix 1'.

If you have more than one appendix, use \verb+\appendices+
\index{commands!appendices@\verb+\appendices+}
and \verb+\include{}+ to include your appendices the same way as chapters.
Each appendix should start with \verb+\chapter{}+, which will label them `Appendix A', `Appendix B', \ldots.

\section{End Matter} % ========================================================

After the body text and appendices come the bibliography, index (optional), and vita, in that order.

\subsection{Bibliography} % ---------------------------------------------------
\label{section:bibliography}

We recommend using BiB\TeX{} to track your references.
\index{BiBTeX@BiB\TeX{}}%
Enter reference information in \texttt{references.bib} and use the \verb+\cite{}+ command to refer to them within the body.
\index{commands!cite@\verb+\cite{}+}
The following commands then create the bibliography.
\begin{verbatim}
    \bibliographystyle{plain}
    \bibliography{references}
\end{verbatim}
\index{commands!bibliography@\verb+\bibliography{}+}
If you have no references\footnote{If there are no references, is it even a dissertation?} and hence \verb+\cite{}+ is never used, comment these commands out to exclude the bibiliography.
On the other end of the spectrum, you can use the command \verb+\nocite{*}+ to print all entries in \texttt{references.bib} in the bibliography, whether or not they were actually cited.

\subsection{Index} % ----------------------------------------------------------
\label{sec:usage:index}

An index is \emph{optional}.
This template uses three ingredients to make an index:
\begin{enumerate}
    \item \verb+\usepackage{makeidx}+ in \texttt{config.tex}.
    \item \verb+\printindex+ in \texttt{main.tex}.
    \item \verb+\index{}+ commands throughout the text.
\end{enumerate}
The \verb+\index{}+ command can be used a few ways.
\index{commands!index@\verb+\index{}+}
\begin{itemize}
    \item \verb+\index{name}+ creates an entry in the index called `name'.
    \item \verb+\index{name@\texttt{label}}+ creates an entry in the index calle `\texttt{label}', placed where `name' should appear.
    \item \verb+\index{list!name}+ creates an entry called `list' and a subentry beneath it called `name'.
\end{itemize}
If you don't have an index, or if you start placing \verb+\index{}+ commands and run out of steam, simply comment out \verb+\printindex+ to remove the index (don't spend the time deleting old \verb+\index{}+ commands, ain't nobody got time for that).
\index{commands!printindex@\verb+\printindex+}

\subsection{Vita} % -----------------------------------------------------------

The vita is a brief biographical sketch of the author and should be written in \texttt{vita.tex}.
Note that the \verb"vita" environment (defined in \texttt{utdiss.sty}) uses the info in \verb+\address{}+ and \verb+\typist{}+ from the preamble to automatically generate a few final lines.
\index{commands!address@\verb+\address{}+}
\index{commands!typist@\verb+\typist{}+}

    \chapter{Examples of Content}
\label{chapter:examples}

This chapter shows some simple examples for including technical information in the body of the report via mathematics, tables, figures, and references.
If you are unfamiliar with \LaTeX{}, see \url{https://www.overleaf.com/learn/latex/Tutorials} for an excellent suite of tutorials.

\section{Writing Mathematics} % ===============================================

Here we show some examples of writing mathematics in the report and using theorem environments.

Here are some equations using the \texttt{align} environment.
\begin{align}
    \label{eq:quadratic}
    0
    &= ax^{2} + bx + c
    \\
    \label{eq:quadraticroots}
    x
    &= \frac{-b + \sqrt{b^{2} - 4ac}}{2a}
\end{align}
\index{environments!align@\verb+align+}
You can refer to equations, like \cref{eq:quadratic}, with \verb+\cref{}+.

The configuration file \texttt{config.tex} defines the following ``theorem'' environments.
\begin{itemize}
    \item \texttt{theorem}
    \index{environments!theorem@\verb+theorem+}
    \item \texttt{corollary}
    \index{environments!corollary@\verb+corollary+}
    \item \texttt{lemma}
    \index{environments!lemma@\verb+lemma+}
    \item \texttt{proposition}
    \index{environments!proposition@\verb+proposition+}
    \item \texttt{definition}
    \index{environments!definition@\verb+definition+}
    \item \texttt{remark}
    \index{environments!remark@\verb+remark+}
    \item \texttt{notation}
    \index{environments!notation@\verb+notation+}
\end{itemize}
The \texttt{proof} environment is also available by default.
\index{environments!proof@\verb+proof+}

\begin{lemma}[An Example Lemma]
\label{lemma:example}

Let $A$ and $B$ be sets with $A \subset B$.
Then $A\cap B = A$.

\begin{proof}
We must show that $A\cap B \subset A$ and $A \subset A \cap B$.
For the first, let $x \in A \cap B$.
By definition, $x \in A$, hence $A \cap B \subset A$.
For the other direction, let $y \in A$. Since $A \subset B$, $y \in B$ as well.
Thus, $y \in A \cap B$, which implies $A\cap B \subset A$.
\end{proof}
\end{lemma}

\begin{theorem}
\label{thm:example}
Theorems look just like lemmata, except they are called `Theorem' instead of `Lemma'.
\end{theorem}

\section{Making Tables} % =====================================================

The \verb"table" environment creates a table, which usually contains a \verb"tabular" environment that houses the actual table text.
The \verb"table" environment takes an optional argument to indicate the position for the table as described in \Cref{table:positions}.
\index{environments!table@\verb+table+}%
\index{environments!tabular@\verb+tabular+}%


\begin{table}[ht]
\centering
\begin{tabular}{r|l}
    argument & position
    \\ \hline
    \texttt{h} & ``here'' if possible \\
    \texttt{t} & top of the page \\
    \texttt{b} & bottom of the page \\
    \texttt{p} & on the page of floats \\
    \texttt{H} & ``here'' no matter what (requires \verb"\usepackage{float}")
\end{tabular}
\caption[Arguments for the \texttt{table} environment.]{
    Arguments for the \texttt{table} environment. It is possible to combine several of these arguments, such as \texttt{ht} (``here'' if possible, otherwise on top of the page). The default is \texttt{tbp}.
}
\label{table:positions}
\end{table}

Use \verb"\caption[short caption]{long caption}" to add a caption to the table.
\index{caption@\verb+\caption{}+}
The \texttt{short caption} is used in the List of Tables
\index{List of Tables@\emph{List of Tables}}%
while the \texttt{long caption} appears in the actual text.
Note that \verb"\label{}" must come after \verb"\caption{}" to create a label that can be referred to elsewhere in the text with \verb"\Cref{}".

\section{Including Figures} % =================================================

The \verb"figure" environment
\index{environments!figure@\verb+figure+}
creates a figure, which usually contains an \verb"\includegraphics{}" command to import the figure from an external file.
The \verb"figure" environment takes the same optional arguments as the \verb"table" environment shown in \Cref{table:positions}.

\begin{figure}[th]
    \centering
    \includegraphics[width=.7\textwidth]{figures/oden.pdf}
    \caption[An example figure.]{This is an example of a figure.}
    \label{figure:example}
\end{figure}

Use \verb"\caption[short caption]{long caption}" to add a caption to the figure.
\index{caption@\verb+\caption{}+}
The \texttt{short caption} is used in the List of Figures
\index{List of Figures@\emph{List of Figures}}%
while the \texttt{long caption} appears in the actual text.
Note that \verb"\label{}" must come after \verb"\caption{}" to create a label that can be referred to elsewhere in the text with \verb"\Cref{}".

We recommend placing figure files in a separate \verb"figures/" folder to keep your working directory organized.


\section{Citations} % =========================================================

The bibliography is discussed in \Cref{section:bibliography}.
Use the \verb"\cite{}" command to refer to entries defined in \texttt{references.bib}:
\begin{itemize}
    \item Cite a single reference \cite{knuth1984texbook}.
    \item Cite multiple references \cite{lamport1994latex, goosens1994latex}.
\end{itemize}

    \include{chapter-conclusion}
\end{verbatim}
\index{commands!include@\verb+\include{}+}%

\subsection{Chapters} % -------------------------------------------------------

The beginning of each chapter file should begin with a \verb+\chapter[]{}+
\index{commands!chapter@\verb+\chapter{}+}%
command, which is similar to \verb"section{}".
The part in \verb"[]" will be listed in the table of contents; the part in \verb"{}" will be printed in the body of the document.
Use \verb+\chapter{}+ (without \verb"[]") to use the same title for both the table of contents and the body.

If your chapter heading consists of more than one line, it will be automatically broken into separate lines.
If you don't like the way LaTeX breaks the chapter heading into lines, use \verb+\newheadline+ to break lines.
\textbf{Never use} \verb+\\+ \textbf{in sectional headings} (\verb+\chapter{}+, \verb+\section{}+, \verb+\subsection{}+, etc.)!
\index{commands!chapter@\verb+\chapter{}+}%
\index{commands!section@\verb+\section{}+}%
\index{commands!subsection@\verb+\subsection{}+}%

\subsection{Appendices} % -----------------------------------------------------
\index{Appendices@\emph{Appendices}}%

If you have no appendices, then proceed to the end matter.

If you have a single appendix, use \verb+\appendix+ and \textbf{do not use}
\verb+\chapter{}+ at the start of the appendix text.
\index{commands!appendix@\verb+\appendix+}
Otherwise the template will insert an extra page with only the word
`Appendix' on it and cause the actual appendix to be labeled `Appendix 1'.

If you have more than one appendix, use \verb+\appendices+
\index{commands!appendices@\verb+\appendices+}
and \verb+\include{}+ to include your appendices the same way as chapters.
Each appendix should start with \verb+\chapter{}+, which will label them `Appendix A', `Appendix B', \ldots.

\section{End Matter} % ========================================================

After the body text and appendices come the bibliography, index (optional), and vita, in that order.

\subsection{Bibliography} % ---------------------------------------------------
\label{section:bibliography}

We recommend using BiB\TeX{} to track your references.
\index{BiBTeX@BiB\TeX{}}%
Enter reference information in \texttt{references.bib} and use the \verb+\cite{}+ command to refer to them within the body.
\index{commands!cite@\verb+\cite{}+}
The following commands then create the bibliography.
\begin{verbatim}
    \bibliographystyle{plain}
    \bibliography{references}
\end{verbatim}
\index{commands!bibliography@\verb+\bibliography{}+}
If you have no references\footnote{If there are no references, is it even a dissertation?} and hence \verb+\cite{}+ is never used, comment these commands out to exclude the bibiliography.
On the other end of the spectrum, you can use the command \verb+\nocite{*}+ to print all entries in \texttt{references.bib} in the bibliography, whether or not they were actually cited.

\subsection{Index} % ----------------------------------------------------------
\label{sec:usage:index}

An index is \emph{optional}.
This template uses three ingredients to make an index:
\begin{enumerate}
    \item \verb+\usepackage{makeidx}+ in \texttt{config.tex}.
    \item \verb+\printindex+ in \texttt{main.tex}.
    \item \verb+\index{}+ commands throughout the text.
\end{enumerate}
The \verb+\index{}+ command can be used a few ways.
\index{commands!index@\verb+\index{}+}
\begin{itemize}
    \item \verb+\index{name}+ creates an entry in the index called `name'.
    \item \verb+\index{name@\texttt{label}}+ creates an entry in the index calle `\texttt{label}', placed where `name' should appear.
    \item \verb+\index{list!name}+ creates an entry called `list' and a subentry beneath it called `name'.
\end{itemize}
If you don't have an index, or if you start placing \verb+\index{}+ commands and run out of steam, simply comment out \verb+\printindex+ to remove the index (don't spend the time deleting old \verb+\index{}+ commands, ain't nobody got time for that).
\index{commands!printindex@\verb+\printindex+}

\subsection{Vita} % -----------------------------------------------------------

The vita is a brief biographical sketch of the author and should be written in \texttt{vita.tex}.
Note that the \verb"vita" environment (defined in \texttt{utdiss.sty}) uses the info in \verb+\address{}+ and \verb+\typist{}+ from the preamble to automatically generate a few final lines.
\index{commands!address@\verb+\address{}+}
\index{commands!typist@\verb+\typist{}+}

    \chapter{Examples of Content}
\label{chapter:examples}

This chapter shows some simple examples for including technical information in the body of the report via mathematics, tables, figures, and references.
If you are unfamiliar with \LaTeX{}, see \url{https://www.overleaf.com/learn/latex/Tutorials} for an excellent suite of tutorials.

\section{Writing Mathematics} % ===============================================

Here we show some examples of writing mathematics in the report and using theorem environments.

Here are some equations using the \texttt{align} environment.
\begin{align}
    \label{eq:quadratic}
    0
    &= ax^{2} + bx + c
    \\
    \label{eq:quadraticroots}
    x
    &= \frac{-b + \sqrt{b^{2} - 4ac}}{2a}
\end{align}
\index{environments!align@\verb+align+}
You can refer to equations, like \cref{eq:quadratic}, with \verb+\cref{}+.

The configuration file \texttt{config.tex} defines the following ``theorem'' environments.
\begin{itemize}
    \item \texttt{theorem}
    \index{environments!theorem@\verb+theorem+}
    \item \texttt{corollary}
    \index{environments!corollary@\verb+corollary+}
    \item \texttt{lemma}
    \index{environments!lemma@\verb+lemma+}
    \item \texttt{proposition}
    \index{environments!proposition@\verb+proposition+}
    \item \texttt{definition}
    \index{environments!definition@\verb+definition+}
    \item \texttt{remark}
    \index{environments!remark@\verb+remark+}
    \item \texttt{notation}
    \index{environments!notation@\verb+notation+}
\end{itemize}
The \texttt{proof} environment is also available by default.
\index{environments!proof@\verb+proof+}

\begin{lemma}[An Example Lemma]
\label{lemma:example}

Let $A$ and $B$ be sets with $A \subset B$.
Then $A\cap B = A$.

\begin{proof}
We must show that $A\cap B \subset A$ and $A \subset A \cap B$.
For the first, let $x \in A \cap B$.
By definition, $x \in A$, hence $A \cap B \subset A$.
For the other direction, let $y \in A$. Since $A \subset B$, $y \in B$ as well.
Thus, $y \in A \cap B$, which implies $A\cap B \subset A$.
\end{proof}
\end{lemma}

\begin{theorem}
\label{thm:example}
Theorems look just like lemmata, except they are called `Theorem' instead of `Lemma'.
\end{theorem}

\section{Making Tables} % =====================================================

The \verb"table" environment creates a table, which usually contains a \verb"tabular" environment that houses the actual table text.
The \verb"table" environment takes an optional argument to indicate the position for the table as described in \Cref{table:positions}.
\index{environments!table@\verb+table+}%
\index{environments!tabular@\verb+tabular+}%


\begin{table}[ht]
\centering
\begin{tabular}{r|l}
    argument & position
    \\ \hline
    \texttt{h} & ``here'' if possible \\
    \texttt{t} & top of the page \\
    \texttt{b} & bottom of the page \\
    \texttt{p} & on the page of floats \\
    \texttt{H} & ``here'' no matter what (requires \verb"\usepackage{float}")
\end{tabular}
\caption[Arguments for the \texttt{table} environment.]{
    Arguments for the \texttt{table} environment. It is possible to combine several of these arguments, such as \texttt{ht} (``here'' if possible, otherwise on top of the page). The default is \texttt{tbp}.
}
\label{table:positions}
\end{table}

Use \verb"\caption[short caption]{long caption}" to add a caption to the table.
\index{caption@\verb+\caption{}+}
The \texttt{short caption} is used in the List of Tables
\index{List of Tables@\emph{List of Tables}}%
while the \texttt{long caption} appears in the actual text.
Note that \verb"\label{}" must come after \verb"\caption{}" to create a label that can be referred to elsewhere in the text with \verb"\Cref{}".

\section{Including Figures} % =================================================

The \verb"figure" environment
\index{environments!figure@\verb+figure+}
creates a figure, which usually contains an \verb"\includegraphics{}" command to import the figure from an external file.
The \verb"figure" environment takes the same optional arguments as the \verb"table" environment shown in \Cref{table:positions}.

\begin{figure}[th]
    \centering
    \includegraphics[width=.7\textwidth]{figures/oden.pdf}
    \caption[An example figure.]{This is an example of a figure.}
    \label{figure:example}
\end{figure}

Use \verb"\caption[short caption]{long caption}" to add a caption to the figure.
\index{caption@\verb+\caption{}+}
The \texttt{short caption} is used in the List of Figures
\index{List of Figures@\emph{List of Figures}}%
while the \texttt{long caption} appears in the actual text.
Note that \verb"\label{}" must come after \verb"\caption{}" to create a label that can be referred to elsewhere in the text with \verb"\Cref{}".

We recommend placing figure files in a separate \verb"figures/" folder to keep your working directory organized.


\section{Citations} % =========================================================

The bibliography is discussed in \Cref{section:bibliography}.
Use the \verb"\cite{}" command to refer to entries defined in \texttt{references.bib}:
\begin{itemize}
    \item Cite a single reference \cite{knuth1984texbook}.
    \item Cite multiple references \cite{lamport1994latex, goosens1994latex}.
\end{itemize}

    \include{chapter-conclusion}
\end{verbatim}
\index{commands!include@\verb+\include{}+}%

\subsection{Chapters} % -------------------------------------------------------

The beginning of each chapter file should begin with a \verb+\chapter[]{}+
\index{commands!chapter@\verb+\chapter{}+}%
command, which is similar to \verb"section{}".
The part in \verb"[]" will be listed in the table of contents; the part in \verb"{}" will be printed in the body of the document.
Use \verb+\chapter{}+ (without \verb"[]") to use the same title for both the table of contents and the body.

If your chapter heading consists of more than one line, it will be automatically broken into separate lines.
If you don't like the way LaTeX breaks the chapter heading into lines, use \verb+\newheadline+ to break lines.
\textbf{Never use} \verb+\\+ \textbf{in sectional headings} (\verb+\chapter{}+, \verb+\section{}+, \verb+\subsection{}+, etc.)!
\index{commands!chapter@\verb+\chapter{}+}%
\index{commands!section@\verb+\section{}+}%
\index{commands!subsection@\verb+\subsection{}+}%

\subsection{Appendices} % -----------------------------------------------------
\index{Appendices@\emph{Appendices}}%

If you have no appendices, then proceed to the end matter.

If you have a single appendix, use \verb+\appendix+ and \textbf{do not use}
\verb+\chapter{}+ at the start of the appendix text.
\index{commands!appendix@\verb+\appendix+}
Otherwise the template will insert an extra page with only the word
`Appendix' on it and cause the actual appendix to be labeled `Appendix 1'.

If you have more than one appendix, use \verb+\appendices+
\index{commands!appendices@\verb+\appendices+}
and \verb+\include{}+ to include your appendices the same way as chapters.
Each appendix should start with \verb+\chapter{}+, which will label them `Appendix A', `Appendix B', \ldots.

\section{End Matter} % ========================================================

After the body text and appendices come the bibliography, index (optional), and vita, in that order.

\subsection{Bibliography} % ---------------------------------------------------
\label{section:bibliography}

We recommend using BiB\TeX{} to track your references.
\index{BiBTeX@BiB\TeX{}}%
Enter reference information in \texttt{references.bib} and use the \verb+\cite{}+ command to refer to them within the body.
\index{commands!cite@\verb+\cite{}+}
The following commands then create the bibliography.
\begin{verbatim}
    \bibliographystyle{plain}
    \bibliography{references}
\end{verbatim}
\index{commands!bibliography@\verb+\bibliography{}+}
If you have no references\footnote{If there are no references, is it even a dissertation?} and hence \verb+\cite{}+ is never used, comment these commands out to exclude the bibiliography.
On the other end of the spectrum, you can use the command \verb+\nocite{*}+ to print all entries in \texttt{references.bib} in the bibliography, whether or not they were actually cited.

\subsection{Index} % ----------------------------------------------------------
\label{sec:usage:index}

An index is \emph{optional}.
This template uses three ingredients to make an index:
\begin{enumerate}
    \item \verb+\usepackage{makeidx}+ in \texttt{config.tex}.
    \item \verb+\printindex+ in \texttt{main.tex}.
    \item \verb+\index{}+ commands throughout the text.
\end{enumerate}
The \verb+\index{}+ command can be used a few ways.
\index{commands!index@\verb+\index{}+}
\begin{itemize}
    \item \verb+\index{name}+ creates an entry in the index called `name'.
    \item \verb+\index{name@\texttt{label}}+ creates an entry in the index calle `\texttt{label}', placed where `name' should appear.
    \item \verb+\index{list!name}+ creates an entry called `list' and a subentry beneath it called `name'.
\end{itemize}
If you don't have an index, or if you start placing \verb+\index{}+ commands and run out of steam, simply comment out \verb+\printindex+ to remove the index (don't spend the time deleting old \verb+\index{}+ commands, ain't nobody got time for that).
\index{commands!printindex@\verb+\printindex+}

\subsection{Vita} % -----------------------------------------------------------

The vita is a brief biographical sketch of the author and should be written in \texttt{vita.tex}.
Note that the \verb"vita" environment (defined in \texttt{utdiss.sty}) uses the info in \verb+\address{}+ and \verb+\typist{}+ from the preamble to automatically generate a few final lines.
\index{commands!address@\verb+\address{}+}
\index{commands!typist@\verb+\typist{}+}

    \chapter{Examples of Content}
\label{chapter:examples}

This chapter shows some simple examples for including technical information in the body of the report via mathematics, tables, figures, and references.
If you are unfamiliar with \LaTeX{}, see \url{https://www.overleaf.com/learn/latex/Tutorials} for an excellent suite of tutorials.

\section{Writing Mathematics} % ===============================================

Here we show some examples of writing mathematics in the report and using theorem environments.

Here are some equations using the \texttt{align} environment.
\begin{align}
    \label{eq:quadratic}
    0
    &= ax^{2} + bx + c
    \\
    \label{eq:quadraticroots}
    x
    &= \frac{-b + \sqrt{b^{2} - 4ac}}{2a}
\end{align}
\index{environments!align@\verb+align+}
You can refer to equations, like \cref{eq:quadratic}, with \verb+\cref{}+.

The configuration file \texttt{config.tex} defines the following ``theorem'' environments.
\begin{itemize}
    \item \texttt{theorem}
    \index{environments!theorem@\verb+theorem+}
    \item \texttt{corollary}
    \index{environments!corollary@\verb+corollary+}
    \item \texttt{lemma}
    \index{environments!lemma@\verb+lemma+}
    \item \texttt{proposition}
    \index{environments!proposition@\verb+proposition+}
    \item \texttt{definition}
    \index{environments!definition@\verb+definition+}
    \item \texttt{remark}
    \index{environments!remark@\verb+remark+}
    \item \texttt{notation}
    \index{environments!notation@\verb+notation+}
\end{itemize}
The \texttt{proof} environment is also available by default.
\index{environments!proof@\verb+proof+}

\begin{lemma}[An Example Lemma]
\label{lemma:example}

Let $A$ and $B$ be sets with $A \subset B$.
Then $A\cap B = A$.

\begin{proof}
We must show that $A\cap B \subset A$ and $A \subset A \cap B$.
For the first, let $x \in A \cap B$.
By definition, $x \in A$, hence $A \cap B \subset A$.
For the other direction, let $y \in A$. Since $A \subset B$, $y \in B$ as well.
Thus, $y \in A \cap B$, which implies $A\cap B \subset A$.
\end{proof}
\end{lemma}

\begin{theorem}
\label{thm:example}
Theorems look just like lemmata, except they are called `Theorem' instead of `Lemma'.
\end{theorem}

\section{Making Tables} % =====================================================

The \verb"table" environment creates a table, which usually contains a \verb"tabular" environment that houses the actual table text.
The \verb"table" environment takes an optional argument to indicate the position for the table as described in \Cref{table:positions}.
\index{environments!table@\verb+table+}%
\index{environments!tabular@\verb+tabular+}%


\begin{table}[ht]
\centering
\begin{tabular}{r|l}
    argument & position
    \\ \hline
    \texttt{h} & ``here'' if possible \\
    \texttt{t} & top of the page \\
    \texttt{b} & bottom of the page \\
    \texttt{p} & on the page of floats \\
    \texttt{H} & ``here'' no matter what (requires \verb"\usepackage{float}")
\end{tabular}
\caption[Arguments for the \texttt{table} environment.]{
    Arguments for the \texttt{table} environment. It is possible to combine several of these arguments, such as \texttt{ht} (``here'' if possible, otherwise on top of the page). The default is \texttt{tbp}.
}
\label{table:positions}
\end{table}

Use \verb"\caption[short caption]{long caption}" to add a caption to the table.
\index{caption@\verb+\caption{}+}
The \texttt{short caption} is used in the List of Tables
\index{List of Tables@\emph{List of Tables}}%
while the \texttt{long caption} appears in the actual text.
Note that \verb"\label{}" must come after \verb"\caption{}" to create a label that can be referred to elsewhere in the text with \verb"\Cref{}".

\section{Including Figures} % =================================================

The \verb"figure" environment
\index{environments!figure@\verb+figure+}
creates a figure, which usually contains an \verb"\includegraphics{}" command to import the figure from an external file.
The \verb"figure" environment takes the same optional arguments as the \verb"table" environment shown in \Cref{table:positions}.

\begin{figure}[th]
    \centering
    \includegraphics[width=.7\textwidth]{figures/oden.pdf}
    \caption[An example figure.]{This is an example of a figure.}
    \label{figure:example}
\end{figure}

Use \verb"\caption[short caption]{long caption}" to add a caption to the figure.
\index{caption@\verb+\caption{}+}
The \texttt{short caption} is used in the List of Figures
\index{List of Figures@\emph{List of Figures}}%
while the \texttt{long caption} appears in the actual text.
Note that \verb"\label{}" must come after \verb"\caption{}" to create a label that can be referred to elsewhere in the text with \verb"\Cref{}".

We recommend placing figure files in a separate \verb"figures/" folder to keep your working directory organized.


\section{Citations} % =========================================================

The bibliography is discussed in \Cref{section:bibliography}.
Use the \verb"\cite{}" command to refer to entries defined in \texttt{references.bib}:
\begin{itemize}
    \item Cite a single reference \cite{knuth1984texbook}.
    \item Cite multiple references \cite{lamport1994latex, goosens1994latex}.
\end{itemize}

    \include{chapter-conclusion}
\end{verbatim}
\index{commands!include@\verb+\include{}+}%

\subsection{Chapters} % -------------------------------------------------------

The beginning of each chapter file should begin with a \verb+\chapter[]{}+
\index{commands!chapter@\verb+\chapter{}+}%
command, which is similar to \verb"section{}".
The part in \verb"[]" will be listed in the table of contents; the part in \verb"{}" will be printed in the body of the document.
Use \verb+\chapter{}+ (without \verb"[]") to use the same title for both the table of contents and the body.

If your chapter heading consists of more than one line, it will be automatically broken into separate lines.
If you don't like the way LaTeX breaks the chapter heading into lines, use \verb+\newheadline+ to break lines.
\textbf{Never use} \verb+\\+ \textbf{in sectional headings} (\verb+\chapter{}+, \verb+\section{}+, \verb+\subsection{}+, etc.)!
\index{commands!chapter@\verb+\chapter{}+}%
\index{commands!section@\verb+\section{}+}%
\index{commands!subsection@\verb+\subsection{}+}%

\subsection{Appendices} % -----------------------------------------------------
\index{Appendices@\emph{Appendices}}%

If you have no appendices, then proceed to the end matter.

If you have a single appendix, use \verb+\appendix+ and \textbf{do not use}
\verb+\chapter{}+ at the start of the appendix text.
\index{commands!appendix@\verb+\appendix+}
Otherwise the template will insert an extra page with only the word
`Appendix' on it and cause the actual appendix to be labeled `Appendix 1'.

If you have more than one appendix, use \verb+\appendices+
\index{commands!appendices@\verb+\appendices+}
and \verb+\include{}+ to include your appendices the same way as chapters.
Each appendix should start with \verb+\chapter{}+, which will label them `Appendix A', `Appendix B', \ldots.

\section{End Matter} % ========================================================

After the body text and appendices come the bibliography, index (optional), and vita, in that order.

\subsection{Bibliography} % ---------------------------------------------------
\label{section:bibliography}

We recommend using BiB\TeX{} to track your references.
\index{BiBTeX@BiB\TeX{}}%
Enter reference information in \texttt{references.bib} and use the \verb+\cite{}+ command to refer to them within the body.
\index{commands!cite@\verb+\cite{}+}
The following commands then create the bibliography.
\begin{verbatim}
    \bibliographystyle{plain}
    \bibliography{references}
\end{verbatim}
\index{commands!bibliography@\verb+\bibliography{}+}
If you have no references\footnote{If there are no references, is it even a dissertation?} and hence \verb+\cite{}+ is never used, comment these commands out to exclude the bibiliography.
On the other end of the spectrum, you can use the command \verb+\nocite{*}+ to print all entries in \texttt{references.bib} in the bibliography, whether or not they were actually cited.

\subsection{Index} % ----------------------------------------------------------
\label{sec:usage:index}

An index is \emph{optional}.
This template uses three ingredients to make an index:
\begin{enumerate}
    \item \verb+\usepackage{makeidx}+ in \texttt{config.tex}.
    \item \verb+\printindex+ in \texttt{main.tex}.
    \item \verb+\index{}+ commands throughout the text.
\end{enumerate}
The \verb+\index{}+ command can be used a few ways.
\index{commands!index@\verb+\index{}+}
\begin{itemize}
    \item \verb+\index{name}+ creates an entry in the index called `name'.
    \item \verb+\index{name@\texttt{label}}+ creates an entry in the index calle `\texttt{label}', placed where `name' should appear.
    \item \verb+\index{list!name}+ creates an entry called `list' and a subentry beneath it called `name'.
\end{itemize}
If you don't have an index, or if you start placing \verb+\index{}+ commands and run out of steam, simply comment out \verb+\printindex+ to remove the index (don't spend the time deleting old \verb+\index{}+ commands, ain't nobody got time for that).
\index{commands!printindex@\verb+\printindex+}

\subsection{Vita} % -----------------------------------------------------------

The vita is a brief biographical sketch of the author and should be written in \texttt{vita.tex}.
Note that the \verb"vita" environment (defined in \texttt{utdiss.sty}) uses the info in \verb+\address{}+ and \verb+\typist{}+ from the preamble to automatically generate a few final lines.
\index{commands!address@\verb+\address{}+}
\index{commands!typist@\verb+\typist{}+}
