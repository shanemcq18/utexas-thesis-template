\chapter{Examples of Content}
\label{chapter:examples}

This chapter shows some simple examples for including technical information in the body of the report via mathematics, tables, figures, and references.
If you are unfamiliar with \LaTeX{}, see \url{https://www.overleaf.com/learn/latex/Tutorials} for an excellent suite of tutorials.

\section{Writing Mathematics} % ===============================================

Here we show some examples of writing mathematics in the report and using theorem environments.

Here are some equations using the \texttt{align} environment.
\begin{align}
    \label{eq:quadratic}
    0
    &= ax^{2} + bx + c
    \\
    \label{eq:quadraticroots}
    x
    &= \frac{-b + \sqrt{b^{2} - 4ac}}{2a}
\end{align}
\index{environments!align@\verb+align+}
You can refer to equations, like \cref{eq:quadratic}, with \verb+\cref{}+.

The configuration file \texttt{config.tex} defines the following ``theorem'' environments.
\begin{itemize}
    \item \texttt{theorem}
    \index{environments!theorem@\verb+theorem+}
    \item \texttt{corollary}
    \index{environments!corollary@\verb+corollary+}
    \item \texttt{lemma}
    \index{environments!lemma@\verb+lemma+}
    \item \texttt{proposition}
    \index{environments!proposition@\verb+proposition+}
    \item \texttt{definition}
    \index{environments!definition@\verb+definition+}
    \item \texttt{remark}
    \index{environments!remark@\verb+remark+}
    \item \texttt{notation}
    \index{environments!notation@\verb+notation+}
\end{itemize}
The \texttt{proof} environment is also available by default.
\index{environments!proof@\verb+proof+}

\begin{lemma}[An Example Lemma]
\label{lemma:example}

Let $A$ and $B$ be sets with $A \subset B$.
Then $A\cap B = A$.

\begin{proof}
We must show that $A\cap B \subset A$ and $A \subset A \cap B$.
For the first, let $x \in A \cap B$.
By definition, $x \in A$, hence $A \cap B \subset A$.
For the other direction, let $y \in A$. Since $A \subset B$, $y \in B$ as well.
Thus, $y \in A \cap B$, which implies $A\cap B \subset A$.
\end{proof}
\end{lemma}

\begin{theorem}
\label{thm:example}
Theorems look just like lemmata, except they are called `Theorem' instead of `Lemma'.
\end{theorem}

\section{Making Tables} % =====================================================

The \verb"table" environment creates a table, which usually contains a \verb"tabular" environment that houses the actual table text.
The \verb"table" environment takes an optional argument to indicate the position for the table as described in \Cref{table:positions}.
\index{environments!table@\verb+table+}%
\index{environments!tabular@\verb+tabular+}%


\begin{table}[ht]
\centering
\begin{tabular}{r|l}
    argument & position
    \\ \hline
    \texttt{h} & ``here'' if possible \\
    \texttt{t} & top of the page \\
    \texttt{b} & bottom of the page \\
    \texttt{p} & on the page of floats \\
    \texttt{H} & ``here'' no matter what (requires \verb"\usepackage{float}")
\end{tabular}
\caption[Arguments for the \texttt{table} environment.]{
    Arguments for the \texttt{table} environment. It is possible to combine several of these arguments, such as \texttt{ht} (``here'' if possible, otherwise on top of the page). The default is \texttt{tbp}.
}
\label{table:positions}
\end{table}

Use \verb"\caption[short caption]{long caption}" to add a caption to the table.
\index{caption@\verb+\caption{}+}
The \texttt{short caption} is used in the List of Tables
\index{List of Tables@\emph{List of Tables}}%
while the \texttt{long caption} appears in the actual text.
Note that \verb"\label{}" must come after \verb"\caption{}" to create a label that can be referred to elsewhere in the text with \verb"\Cref{}".

\section{Including Figures} % =================================================

The \verb"figure" environment
\index{environments!figure@\verb+figure+}
creates a figure, which usually contains an \verb"\includegraphics{}" command to import the figure from an external file.
The \verb"figure" environment takes the same optional arguments as the \verb"table" environment shown in \Cref{table:positions}.

\begin{figure}[th]
    \centering
    \includegraphics[width=.7\textwidth]{figures/oden.pdf}
    \caption[An example figure.]{This is an example of a figure.}
    \label{figure:example}
\end{figure}

Use \verb"\caption[short caption]{long caption}" to add a caption to the figure.
\index{caption@\verb+\caption{}+}
The \texttt{short caption} is used in the List of Figures
\index{List of Figures@\emph{List of Figures}}%
while the \texttt{long caption} appears in the actual text.
Note that \verb"\label{}" must come after \verb"\caption{}" to create a label that can be referred to elsewhere in the text with \verb"\Cref{}".

We recommend placing figure files in a separate \verb"figures/" folder to keep your working directory organized.


\section{Citations} % =========================================================

The bibliography is discussed in \Cref{section:bibliography}.
Use the \verb"\cite{}" command to refer to entries defined in \texttt{references.bib}:
\begin{itemize}
    \item Cite a single reference \cite{knuth1984texbook}.
    \item Cite multiple references \cite{lamport1994latex, goosens1994latex}.
\end{itemize}
